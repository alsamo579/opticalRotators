\section*{Results and Discussion}

\subsection*{Transmittance and birefringence of microvaterite spherulites }
	The physical properties of vaterite are of particular interest since these characteristics govern the particle's interaction with incident light. Past studies (vogel, parkin, bormuth) have, to an extent, investigated the birefringence and reflectivity of vaterite microspheres. However,  specific synthesis conditions give rise to different structural and surface morphologies (brecevic). An understanding of how synthesis-specific  reactants influence microvaterite's transmittance and birefringence are crucial in quantifying the incoming photonic angular momentum's contribution to the rotational torque. 
\subsubsection*{Quantifying birefringence}
When light is incident on a birefringent uniaxial crystal of thickness $d$, it experiences
a phase shift dependent the discrepancy between the refractive indices of the ordinary $n_{0}$ and
extraordinary $n_{e}$ axes (Fig. 1(c)). For the most general case, an elliptically polarized beam 
can be expressed in terms of its ellipticity 
$\phi$ and components parallel and perpendicular to the optic axis of the birefringent medium.
\begin{equation}
E=E_{0}e^{i\omega t}[\cos\phi\cos\theta\hat{i}+i\sin\phi\sin\theta\hat{j}]
\end{equation}


Here $\omega$ is the angular frequency of the incoming wave. When
transmitted through the material, the emerging field will experience
a phase shift $kd\Delta n$. The transmitted field $E^{prime}$, 
\begin{equation}
E^{\prime}=E_{0}e^{i\omega t}[e^{ikdn_{e}}\cos\phi\cos\theta\hat{i}+ie^{ikdn_{0}}\sin\phi\sin\theta\hat{j}]
\end{equation}

Integration of the electric field and it's complex conjugate $\boldsymbol{E^{\prime*}}$
over all spacial elements yields the angular momentum $J=[\epsilon/2i\omega]\int d^{3}r\boldsymbol{E^{\prime*}\boldsymbol{E^{\prime}}$.
The subsequent change in angular momentum results in a torque per
unit area, 
\begin{equation}
\tau=\frac{\epsilon}{2\omega}E_{0}^{2}\biggl[-\sin kd(n_{0}-n_{e})\cos2\phi\sin2\theta+[1-\cos kd(n_{0}-n_{e})]\sin2\phi\biggr]
\end{equation}

In this relation, the first term is the torque owing to the plane polarized component of the incident light. This force tends to align the optic axis of the crystal with the electric field. The second term is the torque resulting from the change in the transmitted light's polarization. This spinning torque is a result of the particle gaining angular momentum, as light passing through a medium is phase shifted by $kd\Deltan$.  For plane polarized light carrying no angular momentum, $\phi=0$ or $\pi/2$, the second term vanishes. Under assumption of a no-slip boundary, the torque on a steady-state spinning particle
will be balanced by the viscous forces. In the case of a spherical particle,
$\tau=8\pi\eta r^{3}$ $\Omega$, with $\Omega=2\pi f$, 
\begin{equation}
f(d,\phi)=A\sqrt{[1-\cos kd(n_{0}-n_{e})]^{2}\sin^{2}2\phi-\sin^{2}kd(n_{0}-n_{e})\cos^{2}2\phi}
\end{equation}
$$A=\frac{J\lambda}{16\pi^{2}c\eta d}$$

Vaterite particles act like microscopic waveplates with positive birefringence $n_{e}>n_{0}$. As a consequence, their optic axis is always aligned
perpendicular to the propogation axis of the incoming beam. In this way, a circularly polarized light beam will always experience the particle's optical anisotropy. In the most general case of eliptically polarized light incident on microvaterite , the particle will rotate  so long as the spinning torque is greater
than the alignment torque. The maximum rotation frequency will occur
for $\phi=\pi/4$, corresponding to incident circularly polarized light. Though $\Delta n$ for a single vaterite crystal is $\Delta n=0.1$, the optical properties of microvaterite particles may change due to their polycrystaline structure resulting from the aggregated nanocrystals during synthesis. The birefringence sets the optimal thickness $d$ for a particle to act as a half-wave plate,
\begin{equation}
d=\frac{\lambda}{2\Delta n}
\end{equation}and is a direct indicator of how effective the particle is at altering incident  optical angular momentum.

We experimentally establish deviations of $\Delta n$ by by measuring the rotation frequency as a function of the incident ellipticity of the beam $\phi.$ An example of the dependence of a $5.40 \mu m$ particle's rotational frequency as a function of $\phi$ is shown in Fig.4(b). As the angle of the quarter-wave plate changes, the polarization of the incident light is altered until a critical polarization $\phi_{onset}$ where the particle begins to rotate. The dependence of  $\phi_{onset}$ on $\Delta n$ is obtained by considering the case where the alignment torque in equation (3) is balanced by the spinning torque. This corresponds to a total torque of zero, and occurs when 
\begin{equation}
\phi_{onset}=\frac{\pm(2m+1)\pi\mp kd\Delta n}{4} , \text{  m=0,1,2,3...}
\end{equation}. The dependence of $f$ on $\phi$ is measured for particles of sizes ranging from \textasciitilde 3-6$\mu m$ in size. Each set
of measurements is then fitted by equation(4).The particle thickness $d=2R$ was acquired by analysis of a microscopic image. $d$ was held fixed; $A$ and $\Delta\chi=k(n_{0}-n_{e})$ were allowed to vary. $A$ was found
to differ by at most $18\%$ from the predicted value, with the incident flux measured to be $J=111 MW/m^2$, and the viscosity of $D_{2}O$ $\eta=1.25mPa\cdot s$ at room temperature. We obtained $\phi_{onset}$ from the point where the solid black line crosses zero in Fig. 4(b). These values are plotted in Fig. 4(a), and fit by equation (5) for $\Delta n$. The point at which $\phi_{onset}=0$ corresponds to the case when a particle is the exact thickness of a half-wave plate. This is shown by the dashed line in the case for a single vaterite crystal where $\Delta n=0.1$.  From the best fit of equation (5), we compute $\Delta\bar{n}=0.050\pm 0.002$. This value is slightly smaller to that obtained in previous studies (Citation, parkin). The results indicate that a synthesized microvaterite particle acts as a perfect half-wave plate for a thickness of $d=10.64\pm 0.43 \mu m$.


\subsubsection*{Measuring transmittance}
	It is well known that a focused Gaussian beam can exert scattering and gradient forces on microparticles (Ashkin 1970 PRL). Indeed, the linear momentum carried by propagating laser light is strong enough to translate particles through a medium. The radiation pressure on a particle is dependent on whether the incident photon is reflected or absorbed. We extend these ideas to our apparatus as a means of quantifying the effective attenuation coefficient of vaterite. 
	Unlike in Ashkin’s seminal experiment, our apparatus uses a wide, collimated laser with energy flux $J$. The electromagnetic field propagates through a 100 $\mu m$ thick capillary containing particles of radii $r$  suspended in $D_{2}O$ (Fig. 4(d). When light is incident on particles from below, the corresponding photon pressure exerts a force that opposes gravity. The momentum flux of the photons can be associated with the energy flux of the electromagnetic wave, such that the corresponding average force of a wave incident normally to the particle's surface is given by, 
$$\langle F_{incident}\rangle=\frac{\alpha J\pi r^2}{c}.$$
Here $c$ is the speed of light and $\alpha$ is the reflectivity.
We observe that for certain ratios of $J$ and $r$ , the particles rise and travel through the thickness of the capillary. Since the $Re\ll1 $ for this system, the subsequent upward motion gives rise to Stokesian drag force $6\pi \eta rv$. Balancing the forces on the particle gives,  
\begin{equation}
\frac {\alpha J_{0} }{6c \eta}-\Delta \frac{2}{9}\frac{\rho r}{\eta} = \frac{v}{r}. 
\end{equation}.
By measuring the time it takes for a particle to rise $100 \mu m$, we extract the rise velocity $v$, which is then related to the reflectivity of vaterite.

The dependence of $v/r$ on $r$  is plotted in Fig.4(c). For increasing particle size, we see that $v/r$ decreases, as expected.  Each point correspond to measured $v/r$ for a particle, with solid lines correspond to fits to equation (7). We define the slope $m$ as $$m=\Delta\rho\frac{2}{9}\frac{g}{\eta}$$
For a vaterite particle suspended in $D_2O$, $\rho=2.64 g/cc$, $\eta=1.25mPa\cdot s$, the predicted value of $m=2.7 1/\mu m\cdot s$. A linear fit to the data shows good agreement to this expected value, with deviations of $<$ 15\%. Each set of measurements provides an the intercept $b$, that scales with the incident energy flux. $b$ provides us with a direct measure of $\alpha$. In particular,  
$$ \alpha=\frac{6bc\eta}{J}$$

Averaging over several fluxes, we find the corresponding value for $\alpha$ is 0.06. For a particle to rise with the same velocity, it must  absorb  $2\alpha$, or $12\%$ of the incoming light.  Given an absorption of $12\%$, we note that the temperature of the fluid surrounding
the particle can be approximated to first order by 
\[
T_{s}=\frac{J_{abs}r}{4\kappa(T_{s})}
\]
 where the absorbed flux is $J_{abs}=2\alpha J_{incident}$, and $\kappa(T_{s})$
is the thermal conductivity of heavy water (Le Nelndre et al ). For
$J_{incident}=130 MW/m^{2}$, the surface temperature of a $r=2.5\mu m$
particle rises according to the upperbound set by the thermal conductivity $\approx 0.6\text{}W/^{\circ}Cm$ . In the case of a $12\%$ absorption,
the surface temperature of the particle is expected to rise by $67.5^{\circ}C$,
and the surrounding fluid will begin to approach the boiling point.
In experiments, we do not observe boiling for fluxes of $130MW/m^{2}$
or higher. With this in mind, we conclude that absorption
of IR light must be negligible; to achieve the same photon pressure
microvaterite must reflect $6\%$ of all incident radiation. 

Surprisingly, the reflectivity and dynamics of microvaterite can be accounted for by the laws of geometric optics. In particular, these laws are valid for particle's with size parameter $x=\text{2\ensuremath{\pi r/\lambda}\ensuremath{\gg1}}$, and the phase shift resulting from the refractive index difference
$\rho=2x(n-n^{\prime})\gg1$. In this regime, rays hit the particle
at various angles and undergo reflection, refraction and diffraction,
before exiting with a change in amplitude and relative phases. Indeed,
the Fresnel theory for a normally incident beam on a large sphere
is in good agreement with the observed reflectivity (Van De Hulst,
Sec. 12.4 Fig 39). 
\subsection*{Rotational dynamics of birefringent vaterite particles}

A vaterite particle's near-wall rotational diffusion plays a crucial role in governing its hydrodynamic mobility. We characterize and measure the rotational dynamics of vaterite particles by making use of depolarized light scattering. The rotational diffusion constants for particles ranging  2-4$\mu m$ in size are shown in Fig. 5(a). These measurements are obtained by depolarized light scattering. Fig. 5(b) depicts stills from a video obtained using a  analyzer/polarizer configuration. The scattered field intensity fluctuates as a particle undergoes translational and rotational diffusion under crossed-polarizers. From the intensity signal, the rotational diffusion constant can be extracted by determining the particle's characteristic rotational decorrelation time. We define the intensity correlation function under crossed polarization (PA) as
\begin{equation}
g_{PA}(q,\tau)=\langle E_{PA}^{*}(q,0)E_{PA}(q,\tau)\rangle/\langle|E_{PA}(q)|^{2}\rangle
\end{equation}

Following the theory in (), for random motion of non-interacting small particles with cylindrical optical symmetry in a viscous fluid, we can assume that the translational/rotational
dynamics are uncoupled.  $g_{PA}$ is thus separable into 
$$
g_{PA}^{(1)}=f_{r}(\tau)f_{t}(q,\tau)
$$
 where 
\begin{equation}
f_{t(}(q,\tau)=\exp(-q^{2}D_{t}\tau)$$
$$f_{r}(\tau)=\exp(-6D_{r}\tau)
\end{equation}

$f_{t}$ and $f_{r}$ are the dynamical correlation functions for rotational and translational motion. Because $f_{r}$  is independent of the scattering vector $\vec{q}$, the rotational diffusion  constant $D_{r}$ of a birefringent particle can be extracted by fitting $f_{r}$ to  $g_{PA}$. 

Fig. 5(c) displays a typical intensity autocorrelation function. The resulting $f_{r}(\tau)$ is well fitted by equation (9), indicated by the red curve. From the fit, we obtain a value for 6$D_{r}$, with the resulting distribution plotted in Fig. 5(a) as a function or particle diameter. We find that our measurements are in good agreement for the dynamic rotational behavior of similar sized spheres diffusing in bulk. These theoretical diffusion constants, obtained via the Stokes-Einstein relation, are indicated by the solid curve.

As vaterite's rotational drag coefficent behaves closely to a Stokesian
sphere, we confirm equation (4) and model a particle's angular frequency by equating the viscous torque $8\pi\eta a^{3}\Omega$ with the optical torque $P/ck$. The optical torque is determined by the power $P$ of the incident photon, and the angular velocity is thus,
\[
\Omega=\frac{1}{8\pi\eta a^{3}}\frac{\sigma P}{ck}
\]
Here $\sigma=[1-\cos(4\pi\Delta na/\lambda$){]}, describes the change in polarization of light transmitted through the spherulite. The explicit dependence of the rotation frequency on parameters $a$, $J$ and $\lambda$ is 
\begin{equation}
f=\frac{J\lambda}{32\pi^{2}c\eta a}\bigg(1-\cos(\frac{4\pi\Delta na}{\lambda})\bigg)
\end{equation}

Experimentally, a linear dependence of $f$ on
$J$ is observed across all particle sizes. We plot the measured rotation frequency for three different particles with sizes 2.45$\mu m$, 3.73$\mu m$ and
4.81 $\mu m.$ in Fig. 5(d).  The frequency is computed by observing
the rotation of a particle under cross-polarizers. In Fig. 5(f) a positive-uniaxial microparticle undergoes one counter-clockwise rotation, during which the polarized light illuminating the sample plane is depolarized periodically by the particle. Due to spherical symmetry, there are four depolarizations ("blinks")
per $2\pi$ revolutions. These blinks occur when the linearly polarized light is not incident
along the particle's ordinary or extraordinary optical
axis. The rotation frequency is proportional to the number of blinks
$N$ in time $t$,
\begin{equation}
f=\frac{N}{4t}
\end{equation}
The uncertainty in the rotation frequency is found by $\delta f=\sqrt{(\frac{1}{4t}\delta N)^{2}+(\frac{N}{4t^{2}}\delta t)^{2}}$.
Solid lines in Fig. 5(d) correspond to expected frequencies for $\Delta n=0.1$. Dashed lines correspond to frequencies for the experimentally measured $\Delta n=0.05$. The measured rotation rates are in good agreement with the expected theoretical values. 

Equation (6) predicts a complex relationship of the rotation frequency on particle size. This is in part due to a competition between the birefringence and viscous forces. The $\sim r^{3}$ dependence of the drag force means that microvaterite particles will tend to rotate slower as the particle size increases, while the sinusoidal optical torque competes with drag by modulating the phase shift of transmitted light with increasing radius. The combination of these two effects results in a non-monotonic behavior, which we observe in Fig. 5(e). The measured rotation frequency as a function of size is a fit to equation (6).

From the data in Figs. 5(d-e),  we conclude that if faster rotation
rates are required, then larger fluxes should be used. Alternatively, smaller particles may be beneficial, but only up to a point - an optimal thickness is required for the particle to act as a half-wave plate. Any deviations from this size begin to reduce the change in angular momentum that occurs as a result of the transmitted light being phase shifted by less than $\pi$. At a critical size, the rotation rate of larger particles starts to decrease due to larger drag forces. Ideal particles should have sizes in the $3-4\mu m$ range, for which we observe the maximum rotation frequency.

\subsection*{Translational dynamics in the presence of energy flux}

So that a particle's dynamics can reciprocate from the surrounding fluid,
it is necessary to minimize any trapping forces that may result from gradients
in the beam intensity. In the presence of no radial forces, the translational motion of a particle is expected to be Brownian. We analyze the dynamics of 
vaterite as it rotates in light of varying $J$ (Supplementary movie).
To quantify the average diffusive behavior the conventional approach
is to compute the time-averaged mean squared displacement (MSD),
\begin{equation}
\langle r^{2}(\tau)\rangle=\frac{1}{T-\tau}\int_{0}^{T-\tau}(r_{i}(t+\tau)-r_{i}(t))^{2}dt
\end{equation}
where $\tau$ is the lag time and $T$ is the total measured time.
We used the particle tracking Python library, trackpy, to obtain
MSDs for single-particle trajectories with different $J$. Fig. 6(a) shows the individual particles' trajectories for different fluxes. The motion is largely Brownian, with the amplitude of diffusion appearing to grow with increasing $J$.  We plot the time-averaged MSD for several fluxes in Fig. 6(b). Each MSD is obtained by averaging the individual mean squared displacement over several particle sizes. For a typical diffusive process in 2-dimensions, $\langle r^{2}(\tau)\rangle=4D_{t}t^{\n}$
where $D_{t}$ is the translational diffusion constant and $\n=1$. To measure the diffusive behavior, we fit a power law to the MSDs in Fig. 6(b). The resulting $n$ and $D_{t}$ are plotted in Fig. 6(c) and (d), respectively. These measurements quantitatively verify our observations of translational Brownian fluctuations in Fig. 6(a). For $J=0$, the translational motion of the particles is purely diffusive, as expected. As the incident flux increases, trapping forces is expected to scale with the gradient in the intensity electric field $\vec{E}%2=I(\vec{r}), 
\begin{equation}
\vec{F}\sim\nabla I(\vec{r})
\end{equation}
In the presence of any radial forces arising from gradients in incident field, the particle trajectories would become super-diffusive or ballistic. The resulting motion corresponds to the particle being guided to the minimum energy configuration at the center of the optical trap. As $J$ increases, we continue to observe constant values for $n$ while, indicating that the rotating particle's experience no trapping forces while in the presence of the torque field. Unexpectedly, we notice that the values for $D_{t}$ begin increase. We attribute this discrepancy to changes in the gravitational height $h_{g}$ caused by the incident
radiation pressure (Fig. 3(c)). In particular, 
\begin{equation}
h_{g}=\frac{k_{B}T}{\Delta\rho\frac{4}{3}\pi r^{3}g-\frac{\alpha J\pi r^{2}}{c}}
\end{equation}
When the position of a moving particle with respect to a stationary surface changes, the flow field  is modified. This, in turn,  affects the particle's hydrodynamic drag. We correct for increasing $h_{g}$ by making use of  lubrication theory  (Citation goldman cox brenner, 1967 eq. 25, 2.59). In particular, for a sphere undergoing slow viscous motion parallel to the plane of the wall, the correction to the drag force at the lubrication limit is 
\begin{equation}
F_{\parallel}=6\pi\eta aF^{*}U$$
$$F^{*}=\frac{8}{15}log\bigg(\frac{h_g}{a}\bigg)
\end{equation}
The solid lines in Fig. 6(d) show the corrected diffusion constant plotted for a 2$\mu m$ and 5 $\mu m$ particle, respectively. When accounting for the drag force correction, the observed increasing in $D_{t}$ is within the expected range.