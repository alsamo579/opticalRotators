\section*{Introduction}
Rotating colloids exhibit remarkably diverse behavior reminiscent of the locomotion of waltzing algae (CITATION), and of gears that power a bacterium’s flagella (CITATION). When submersed in a fluid, the hydrodynamic forces between micro-rotators can exert attractive and repulsive interactions of varying range. As such, the directed manipulation of colloidal particles by external fields is of particular interest in the study of dynamic self-assembly, microrheology, and the development of new synthetic, steerable colloids (CITATIONS). 

Traditionally, torques over particulate matter have been exerted by magnets (CITATIONS). When a magnetic colloid is introduced to a time-dependent magnetic field, it will realign so that its moment lies parallel to the field. Consequently, such particles undergo physical motion and experience viscous resistance when they are suspended in a liquid. Magnetic fields present unique advantages in that (a) the frequency and direction of the torque field set the angular velocity of the particle and (b) they are easy to integrate and model due to their unscreened nature in electrolyte solution. Yet, the homogeneity of a magnetic torque field restricts the rotation direction - particles can be rotated exclusively in one direction, either clockwise or counterclockwise.  The aim of this report is to introduce an alternative torque field capable of rotating hundreds of colloidal micro-particles.

By harnessing the spin angular momentum of circularly polarized light, various laser configurations have succeeded in applying torques to optically trapped particles. Such optical micro-manipulation has been achieved using birefringent dielectric particles (CITATION), which are capable of inducing a phase retardation between the ordinary and extraordinary components of the beam. The result is a change in the angular momentum of the transmitted light, corresponding to a torque and subsequent rotation of the particle. Minimum thermal energy is gained by the experimental system due to the transparent nature of the birefrignent particle. This allow for experiments to be conducted at high powers, thereby achieving higher rotation rates (CITATION).

At present, optical torque fields have exclusively rotated particles at the focal point of a highly converging beam. This converging beam geometry applies a strong radial, "tweezing" force (CITATION) that keeps the particle pivoting at a fixed point. As a result, the dynamics of the rotating particle are dictated by the beam, and do not reciprocate from the surroundings. Furthermore, the narrow waist of the focused beam limits the the number of particles that can be rotated to no more than a handful. 

Here we report a tailored optical system capable of inducing stable rotation in hundreds of birefringent particles simultaneously. We demonstrate a protocol for synthesis of stable micro-vaterite particles that can rotate for long periods of times. Using a broad, defocused Gaussian beam, the particles experience a torque field with minimal trapping forces that would otherwise arise from gradients in the light intensity.The translational behavior of rotating particles in the presence of the beam is then analyzed. Altering the handedness and intensity of the laser source allows us to control the rotation frequency and switch between clockwise or counterclockwise rotation.  We characterize vaterite's optical and physical properties through experiment by measuring their transmittance and birefringence. With this information, we establish a dependence of rotation frequency on various parameters, such as particle size, energy flux and the viscous drag force acting on the particle. Our study offers a novel degree of freedom for rotational optical micro-manipulation that we believe can be combined in tandem with magnetic torque fields.
