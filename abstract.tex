Optical angular momentum can be transferred to a particle resulting in a torque that leads to the particle's rotation. In the past, optical tweezers have been used to rotate individual colloids, trapping them to the narrow waist of a focused laser. We show that a defocused, circularly-polarized beam can induce stable rotation in hundreds of colloidal micro-particles simultaneously. This is done by developing a process for synthesizing stable birefringent vaterite, capable of rotating for long periods of time while experiencing minimal tweezing forces. Altering the handedness and intensity of the laser source allows us to control the frequency and switch between clockwise or counterclockwise rotation. We discuss the optical and physical properties of the rotating particles.