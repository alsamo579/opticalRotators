\section*{Materials and Methods}
\subsection*{Synthesis and stabilization of vaterite microspheres}
We synthesize microvaterite particles via controlled precipitation of highly concentrated solutions of calcium chloride $CaCl_{2}$ and sodium carbonate $Na_{2}CO_{3}$ (cite, x,y,z). These solutions are buffered to a pH of 9-9.5 by N-Cyclohexyl-2-aminoethanesulfonic acid (CHES). Initially, equal volumes (5mL) of 0.33 M $Na_{2}CO_{3}$ and $CaCl_2$ are mixed in a glass vial containing a 1cm magnetic stir-bar, rotating at $\approx 950rpm$ Fig. 2(a,1). This sample is referred to as the synthesis bath. Under these conditions, a typical yield results in a cloudy mixture with  thousands of particles with a mean size of 3.6 um $\pm$ 0.8 um (Fig.2(b)). By varying the stirring speed, initial reactant concentration, and reaction time the final diameter of the synthesis bath can be  (volodkin et. al). In this way, we are capable of synthesizing particles ranging from 2-12$\mu m$ in diameter.

A major barrier encountered when resuspending vaterite microspheres in solvent is their sensitivity to pH. Vaterite is a metastable polymorph of calcium carbonate, preserving its phase exclusively in basic conditions. Deviations towards even weakly acidic conditions cause rapid dissolution and transformation of vaterite spheres to calcite cubes (citation). So that we could promote phase stability, limit flocculutation, and preserve vaterite's optomechanical behavior, we choose to coat particles first with (3- Aminopropyl)trimethoxysilane (APS) and tetraethyl orthosilicate (TEOS) Fig. 2(a,2). In a typical APS coating, 1.5mL of the synthesis bath is washed in DI H20 3 times, to which 70uL of APS (Aldrich), 25uL of Ammonia (25\% V/V in H20, Merck) and 940uL of ethanol (200 proof) are added. The sample is then placed in a shaker for 2.5 hours. For TEOS coatings, the procedure is identical, except the sample is allowed to shake for 5 hours (cite Vogel?). 

Scanning electron microscopy (SEM) images at various points of the synthesis process show a polycrystalline amalgamation of nanocrystals combining to form a spheroidal microvaterite particle (Fig. 2(c). Previously, it has been shown that direction of the optical axis resulting from these nanocrystalline subunits is oriented in a hyperbolic distribution throughout the formed microvaterite crystal. This suggests that an effective optical anisotropy exists at the particle scale. (Parkin et al . 2009 optics express).
\subsection*{Optical setup}
Our instrument is designed to apply a uniform optical torque field via delivery of a collimated beam to the sample plane. A schematic diagram of the setup can be seen in Fig. 3(a). The experimental system consists of a 3mm infrared (IR) laser, $\lambda=1064$nm (IPG Photonics, $M^{2}=1$), propagating first through a half wave plate ($\lambda/2$) followed by two Galilean beam contractors. The two telescopes are capable of shrinking the beam by a factor of 30, while preserving the initial collimation of the laser source. Once the beam diameter has been reduced, the laser propagates through a polarizing beam splitter (PBS) cube where the S-component of light is reflected upwards. The handedness of the linearly polarized light can be altered via a quarter wave plate ($\lambda/4$) fixed immediately after the PBS atop a precision manual rotation mount. 

We are able to image the distribution of the incident near-infrared beam by making use of an infrared viewing card. The pixel intensities of the image are then fit to a Gaussian surface of revolution to obtain an intensity distribution of the beam flux at the sample plane. The flux $J$ can be written as 
\begin{equation}
J(\textbf{r})=\frac{I}{\pi\sigma^{2}}exp\bigg[-\frac{r^{2}}{\sigma}\bigg]\end{equation}
where the pre-factor corresponds to the flux at the center of the beam and is obtained by integration of the 2-dimensional Gaussian over the area of the beam.
Fig 3(b) shows the intensity distribution $I(r)$ with $\sigma$=235 $\mu m$. To compensate for any divergence that may result from minimizing the beam diameter, we use a 25.4 mm biconvex lens with a focal length of 15 cm. This compensating lens sits within a non-rotating zoom housing, which allows for fine tuning of the beam diameter. We are capable of adjusting the flux at the sample either by changing the size of the incoming beam, or modulating the power at the laser head. 

To image our microspheres, we illuminate the sample using a 505 nm LED (Thorlabs). The light emitted by the LED passes through a N-BK7 ground glass diffuser (600 grit), and is then pseudo-collimated with an N-BK7 plano-convex lens (f=25.4mm) that acts as a primary collector lens. After a secondary achromatic doublet lens (f=76.2mm) forms an image of the filament at the position of a field diaphragm. A condenser lens is used to focus the light at the back focal plane of the objective. Once Köhler illumination is achieved, images of the focal plane are obtained using an infinity corrected Leica HCX PL APO 40x objective (NA=0.85). High-pass index matched IR filters mounted before the camera are used to prevent interference of IR light with the 3.1MP monochrome camera (Imaging Source). 